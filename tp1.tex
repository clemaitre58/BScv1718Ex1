\documentclass[12pt]{tdtp}
\usepackage{tabularx,colortbl}
\usepackage{multirow}
\usepackage{listings}
\lstset{
	language=VHDL,
basicstyle=\tiny\ttfamily}
\definecolor{light-gray}{gray}{0.96}
\definecolor{pageheading-gray}{gray}{0.2}
\definecolor{dark-gray}{gray}{0.45}
\definecolor{dark-green}{rgb}{0.245,0.121,0.0}

\newcommand{\auteur}{Cedric Lemaitre}
\newcommand{\couriel}{c.lemaitre58@gmail.com}
\newcommand{\promo}{L3 Pro Robotique}
\newcommand{\annee}{2017-2018}
\newcommand{\matiere}{Traitement M3.1}

\newcommand{\tdtp}{TP 1}
\renewcommand{\sujet}{Inititation Python}


\begin{document}
\titre
Ce TP propore un ensemble d'exercices basiques en python en vue des séances prochaines.\\
\\
\\
\\\
\textit{NB : Tous les éxercices doivent utiliser des fonctions de façon à rendre votre code modulaire }

%%%%%%%%%%%
\Exo

Écrire un programme permettant de calculer et d'afficher l'aire et le périmètre d'un cercle dont vous préciserez le rayon.


%%%%%%%%%%%%
\Exo


Écrire une fonction qui détermine et retourne le minimum et le maximum d'un tableau.


%%%%%%%%%%%
\Exo 

Écrire une fonction qui calcule la suite de Fibonacci. Le rang de la suite sera passée en paramêtre.


%%%%%%%%%%
\Exo


Écrire une fonction qui permet d'afficher n'importe quelle table de multiplication dans la console.

%%%%%%%%%%
\Exo

Écrire une fonction qui décompose un entier que vous passerez en paramêtre en valeur binaire. \textit{Par exemple, vous passez 8 en argument et la fonction vous retourne 10}


%%%%%%%%%%
\Exo

Écrire une fonction qui calcule automatiquement la table de vérité des opérateurs logiques : 
\begin{enumerate}
	\item et (and)
	\item ou (or)
	\item n'est pas (not)
\end{enumerate}

La table de vérité sera calculée pour trois variable d'entrée.
%%%%%%%%%

\end{document}
