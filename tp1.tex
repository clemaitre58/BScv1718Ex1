\documentclass[12pt]{tdtp}
\usepackage{tabularx,colortbl}
\usepackage{multirow}
\usepackage{listings}
\lstset{
	language=VHDL,
basicstyle=\tiny\ttfamily}
\definecolor{light-gray}{gray}{0.96}
\definecolor{pageheading-gray}{gray}{0.2}
\definecolor{dark-gray}{gray}{0.45}
\definecolor{dark-green}{rgb}{0.245,0.121,0.0}

\newcommand{\auteur}{Cedric Lemaitre}
\newcommand{\couriel}{c.lemaitre58@gmail.com}
\newcommand{\promo}{BScv }
\newcommand{\annee}{2017-2018}
\newcommand{\matiere}{Computer science}

\newcommand{\tdtp}{Exercise session \#1}
\renewcommand{\sujet}{C++ introduction}


\begin{document}
\titre
You could find here some basic exercises.\\
\\
\\
\\\
\textit{NB : you have to use for each modular code in other words : use function!!!}

%%%%%%%%%%%
\Exo


Write a program which allows to compute Area and circumference of a circle. (Input of the function : circumference)

%%%%%%%%%%%%
\Exo


Write a function which allows to find the max and the min of a table.

%%%%%%%%%%%
\Exo 

Write a function which allows to compute the Fibonacci sequence. (Input of the function : rank of the sequence)

%%%%%%%%%%
\Exo


Write a function which allows to display in the terminal any multiplication table.
%%%%%%%%%%
\Exo

Write a function which allows to compute the truth table of the following operators : 
\begin{enumerate}
	\item and
	\item or
	\item no
\end{enumerate}

each table will be computed using 3 variable as input.

%%%%%%%%%

\end{document}
